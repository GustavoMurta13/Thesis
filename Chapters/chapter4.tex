%!TEX root = ../template.tex
%%%%%%%%%%%%%%%%%%%%%%%%%%%%%%%%%%%%%%%%%%%%%%%%%%%%%%%%%%%%%%%%%%%%
%% chapter4.tex
%% NOVA thesis document file
%%
%% Chapter with lots of dummy text
%%%%%%%%%%%%%%%%%%%%%%%%%%%%%%%%%%%%%%%%%%%%%%%%%%%%%%%%%%%%%%%%%%%%

\typeout{NT FILE chapter4.tex}%

\chapter{Conclusions}

    The gaming industry is rapidly advancing within the entertainment sector, offering significant opportunities for substantial innovation and seamless integration of cutting-edge technologies. And in parallel the streaming industry is flourishing both in revenue and popularity. This growth brings possible problems that can be resolved with innovative technology, such as an automatic event detection system. This system can bring multiple benefits to the streamer who wants to create \gls{VOD}s and to the viewer who wants to watch a specific moment of a video that is too long to browse manually.

    To navigate through the concepts presented in the dissertation, an overview of video games, their evolution, and the optimal platform for video game streaming was introduced in the section labeled \nameref{sec:BackInfo}. In this section, it is described the structure of the platform Twitch and the technology behind the delivery of the broadcast to every viewer.

    To develop an event detection system it was needed to provide an overlook of the current technologies used to detect events in sports videos, as done in Chapter \nameref{cha:StateArt}. All of these technologies consisted of collecting low-level information (features) that alone could not be used to infer key moments of the sports video, denominated of \gls{LLF}, grouping them, and combining them to create models. All these models, being visual, audio models, or text models, are labeled as \gls{MLR}. These models are then processed with domain knowledge regarding the broadcast resulting in high-level information that is regarded as \gls{HLS}.

    In the same chapter, it was given an overlook of tools that are being currently used to detect events and highlights in video games. The section \nameref{sec:EventDetctionVideogames} describes an integrated tool in graphic engines that allows developers to trigger flags when a key moment of the game is played. This technology is another approach to the problem of detecting events in video game feeds.

    Finally, Chapter \nameref{cha:Planning} describes a plan and timeline for the development of this thesis. This timeline describes the tasks and sub-tasks to be made and the period that they should be developed. This timeline is not final and can be modified to correct the road of the development when needed.

    The gaming industry is experiencing substantial growth, and concurrently, the gaming broadcast sector is also expanding rapidly. The adoption of such systems plays a crucial role in enhancing the overall quality of gaming broadcasts. As the industry continues to flourish, the demand for these systems is escalating, making them increasingly indispensable tools for the gaming sector.

    
